% !TeX root = ./main.tex

\ustcsetup{
  title              = {基于乐观锁的内存数据库\\索引模块设计与实现},
  title*             = {Design and Implementation of In-Memory Database Index Module Based on Optimistic Lock},
  author             = {XXX},
  author*            = {XXX},
  speciality         = {软件工程},
  speciality*        = {Software Engineering},
  supervisor         = {XXX~XXX},
  supervisor*        = {XXX},
  % date               = {2017-05-01},  % 默认为今日
  % professional-type  = {专业学位类型},
  % professional-type* = {Professional degree type},
  % department         = {软件学院},  % 院系,本科生需要填写
  % student-id         = {XXX},  % 学号,本科生需要填写
  % secret-level       = {秘密},     % 绝密|机密|秘密|控阅,注释本行则公开
  % secret-level*      = {Secret},  % Top secret | Highly secret | Secret
  % secret-year        = {10},      % 保密/控阅期限
  %
  % 数学字体
  % math-style         = GB,  % 可选:GB, TeX, ISO
  math-font          = xits,  % 可选:stix, xits, libertinus
}


% 加载宏包

% 定理类环境宏包
\usepackage{amsthm}

% 插图
\usepackage{graphicx}

% 绘图
\usepackage{pgfplots}
\usepackage{tikz}
\pgfplotsset{width=6.6cm,compat=1.7}

% 三线表
\usepackage{booktabs}

% 跨页表格
\usepackage{longtable}

% 算法
% \usepackage[ruled,linesnumbered]{algorithm2e}
\usepackage{algorithm}
\usepackage{algorithmic}
\usepackage{float}  
\usepackage{lipsum}
\makeatletter
\newenvironment{breakablealgorithm}
{% \begin{breakablealgorithm}
	\begin{center}
		\refstepcounter{algorithm}% New algorithm
		\hrule height.8pt depth0pt \kern2pt% \@fs@pre for \@fs@ruled
		\renewcommand{\caption}[2][\relax]{% Make a new \caption
			{\raggedright\textbf{\ALG@name~\thealgorithm} ##2\par}%
			\ifx\relax##1\relax % #1 is \relax
			\addcontentsline{loa}{algorithm}{\protect\numberline{\thealgorithm}##2}%
			\else % #1 is not \relax
			\addcontentsline{loa}{algorithm}{\protect\numberline{\thealgorithm}##1}%
			\fi
			\kern2pt\hrule\kern2pt
		}
	}{% \end{breakablealgorithm}
		\kern2pt\hrule\relax% \@fs@post for \@fs@ruled
	\end{center}
}
\makeatother


% SI 量和单位
\usepackage{siunitx}

% 参考文献使用 BibTeX + natbib 宏包
% 顺序编码制
% \usepackage[sort]{natbib}
% \bibliographystyle{ustcthesis-numerical}

% 著者-出版年制
% \usepackage{natbib}
% \bibliographystyle{ustcthesis-authoryear}

% 本科生参考文献的著录格式
% \usepackage[sort]{natbib}
% \bibliographystyle{ustcthesis-bachelor}

% 参考文献使用 BibLaTeX 宏包
\usepackage[style=ustcthesis-numeric]{biblatex}
%\usepackage[bibstyle=ustcthesis-numeric,citestyle=ustcthesis-inline]{biblatex}
%\usepackage[style=ustcthesis-authoryear]{biblatex}
%\usepackage[style=ustcthesis-bachelor]{biblatex}
% 声明 BibLaTeX 的数据库
\addbibresource{bib/ustc.bib}

% 配置图片的默认目录
\graphicspath{{figures/}}

% 数学命令
\makeatletter
\newcommand\dif{%  % 微分符号
  \mathop{}\!%
  \ifustc@math@style@TeX
    d%
  \else
    \mathrm{d}%
  \fi
}
\makeatother
\newcommand\eu{{\symup{e}}}
\newcommand\iu{{\symup{i}}}

% 用于写文档的命令
\DeclareRobustCommand\cs[1]{\texttt{\char`\\#1}}
\DeclareRobustCommand\pkg{\textsf}
\DeclareRobustCommand\file{\nolinkurl}

% hyperref 宏包在最后调用
\usepackage{hyperref}
