% !TeX root = ../main.tex

\chapter{总结与展望}

\section{论文工作总结}
随着计算机硬件的发展,同时也在新的应用的需求推动下,内存数据库成为了相对热门的研究方向。索引作为数据库中至关重要的模块,
决定着数据库的基本性能,本文在总结了前人们的提出的不同的索引技术的基础上,选取相对高性能的内存数据库索引结构作为研究对象。
在内存数据库系统中实现了ART索引,同时为了应对索引的并发操作,引入了基于乐观级联锁的并发控制算法。本文的主要工作包括:

1. 分析了当前内存数据库索引模块的瓶颈,介绍了当前多线程高并发场景下,设计缓存友好型和多核平台上高扩展性索引的必要性

2. 选取ART索引作为B+树索引的替代者,同时使用乐观级联锁处理并发操作,并实现了索引的基本的插入,查询,删除,范围查询等接口
设计不同类型键值的存储方式,设计实现相应的内存管理模块,设计实现了索引对最长前缀匹配的支持等功能模块。

3. 实现了针对该索引的功能性和非功能性测试,主要测试了索引的功能是否完善。性能测试方面主要是针对不同的工作线程,多读少写混合模式下
索引是否具有良好的可扩展性。根据前文实验的结论,该索引的可扩展性获得预期的效果。

总上,基于乐观级联锁的索引模块符合设计的预期,在多读少写的场景下具备完好的可扩展性,同时,利用CPU调度等机制,
避免了极端场景下,乐观锁导致的重复操作的问题。
\section{研究展望}

本文作为内存数据库索引的研究实现中的一个小方向,主要是从可扩展性和缓存友好性考虑,同时需要针对实际的使用场景作出例如支持最长前缀匹配,
支持不同的数据类型。内存数据库的索引机制还可以在以下几个方面继续深入。

1. 对于该内存数据库,后续的计划会将热数据从内存中转换为Apache Arrow的格式,转存到SSD或者磁盘上,方便后续的数据分析相关的工作。此时
数据库索引应该作出相应的修改,尤其是原先只是根据指针作为二级索引的Value。

2. 针对前文中我们提到内存数据库的表存储采用的是PAX行列混存,变长字段可以采用表级的变长字段管理器统一管理,实际表中只存储相应的Tag。
实际上,如果变长字段上有索引,而我们的索引实际上是全量存储的,那么该表上的变长字段就被存储了两份,如果变长字段较长,该部分冗余量较大,会造成索引
内存占用较大,实际上这里也只需要存储索引的一份字段,降低内存开销。


